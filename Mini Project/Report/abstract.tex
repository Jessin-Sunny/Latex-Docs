\newpage
\thispagestyle{empty}
\chapter*{Abstract}
\addcontentsline{toc}{chapter}
{\numberline{}Abstract}%
Package assignment and efficient route planning are critical aspect of logistics and delivery services. Traditional 
methods often result in increased delivery times, fuel consumption, and operational costs. This project aims to develop a system that assigns packages efficiently and calculates the shortest and efficient delivery route for a delivery person by leveraging graph theory principles. The solution addresses the need for optimization in delivery services, reducing travel time and resource utilization while improving customer satisfaction. 
\\
The package assignment process follows an optimized allocation strategy to ensure minimal travel distance and efficient workload distribution. Packages are first assigned to delivery personnel based on their designated quadrant, reducing unnecessary cross-region travel. If a delivery agent has additional capacity, the system dynamically reallocates packages from nearby quadrants to maximize efficiency. Furthermore, the system prioritizes urgent deliveries and adapts to real-time constraints, such as vehicle capacity and traffic conditions, ensuring timely and cost-effective deliveries. If any packages remain unassigned, they are scheduled for the next delivery cycle with high priority.
\\
The proposed system will utilize graph-based algorithms, specifically Greedy Algorithm, to compute the shortest path between multiple delivery locations. Delivery points will be represented as nodes in a graph, and the paths connecting them as weighted edges, where the weights denote distance. The system will 
process input data such as the list of delivery locations and dynamically generate the efficient route for completing all deliveries.
\\
The system consists of a web platform for company operations and a mobile app for delivery personnel. The website, built with HTML, CSS, and JavaScript, enables company registration, employee management, package uploads, and delivery assignments. The backend, developed using Python with Firebase Firestore, handles data storage and synchronization. The mobile app, built with Flutter, provides real-time navigation, package details, and notifications, ensuring seamless tracking and efficient delivery operations.
\\
The expected outcome of this project is a reliable and user-friendly routing system that 
minimizes delivery time and resource usage. The solution demonstrates the practical 
application of graph theory in real-world scenarios and offers significant benefits to logistics 
companies and delivery personnel by enhancing operational efficiency and productivity. 